% Copyright 2016 - 2017 Bas van Meerten and Wouter Franssen
%
%This file is part of ssNake.
%
%ssNake is free software: you can redistribute it and/or modify
%it under the terms of the GNU General Public License as published by
%the Free Software Foundation, either version 3 of the License, or
%(at your option) any later version.
%
%ssNake is distributed in the hope that it will be useful,
%but WITHOUT ANY WARRANTY; without even the implied warranty of
%MERCHANTABILITY or FITNESS FOR A PARTICULAR PURPOSE.  See the
%GNU General Public License for more details.
%
%You should have received a copy of the GNU General Public License
%along with ssNake. If not, see <http://www.gnu.org/licenses/>.

\usepackage[british]{babel}
\usepackage{graphicx,booktabs,listings,amsmath,pgfplots,pgfplotstable}
\usepackage[small,bf,nooneline]{caption}
\usepackage{subcaption}
\usepackage[sort&compress,numbers]{natbib}
\usepackage{tikz}
\usepackage{mathtools}
\usepackage[nottoc]{tocbibind}%adds bibliography to table of contents.
\graphicspath{{./images/}}
%\setlength{\textwidth}{453pt} %597 pt is the a4 paperwidth. Minus 2 in margin. 72 pt = 1 in
%\setlength{\hoffset}{-\oddsidemargin}
%\setlength{\voffset}{-30pt} %
%\setlength{\textheight}{651 pt} %a4 height 845 pt minus 2* total headheight. In this case 2*88pt
%% examine margines via the layout package. Use command \layout{} in document to draw a picture.
%\setlength{\parindent}{0.5 cm}
%\setlength{\parskip}{0 cm}
\usepackage[left=82pt,right=82pt,top=95pt,bottom=95pt,footnotesep=0.5cm]{geometry}
%\setlength{\headheight}{14pt}

%define colours--------------------
%dark
\usepackage{xcolor}
\definecolor{MyGrayD}{RGB}{1,1,1}
\definecolor{MyRedD}{RGB}{237,45,46}
\definecolor{MyGreenD}{RGB}{0,140,71}
\definecolor{MyBlueD}{RGB}{24,89,169}
\definecolor{MyOrangeD}{RGB}{243,125,34}
\definecolor{MyPurpleD}{RGB}{102,44,145}
\definecolor{MyBrownD}{RGB}{161,29,32}
\definecolor{MyPinkD}{RGB}{179,56,147}
%normal
\definecolor{MyGray}{RGB}{114,114,114}
\definecolor{MyRed}{RGB}{241,89,95}
\definecolor{MyGreen}{RGB}{121,195,106}
\definecolor{MyBlue}{RGB}{89,154,211}
\definecolor{MyOrange}{RGB}{249,166,90}
\definecolor{MyPurple}{RGB}{158,102,171}
\definecolor{MyBrown}{RGB}{205,112,88}
\definecolor{MyPink}{RGB}{215,127,179}
%light
\definecolor{MyGrayL}{RGB}{204,204,204}
\definecolor{MyRedL}{RGB}{242,174,172}
\definecolor{MyGreenL}{RGB}{216,228,170}
\definecolor{MyBlueL}{RGB}{184,210,235}
\definecolor{MyOrangeL}{RGB}{242,209,176}
\definecolor{MyPurpleL}{RGB}{212,178,211}
\definecolor{MyBrownL}{RGB}{221,184,169}
\definecolor{MyPinkL}{RGB}{235,191,217}
%----------------------------------

%Figure ref with hyperref
\newcommand{\fref}[1]{\hyperref[#1]{Figure \ref*{#1}}}
\newcommand{\sref}[1]{\hyperref[#1]{Section \ref*{#1}}}
\newcommand{\tref}[1]{\hyperref[#1]{Table \ref*{#1}}}

%Makes a new command for figures with input values: filename, width(times linewidth),
% caption and label.
\newcommand{\onefigure}[4]{
\setlength{\captionwidth}{#2\linewidth}
\begin{figure}
\includegraphics[width=#2\linewidth]{#1}
\centering
\parbox{\linewidth}{\caption{#3}
\label{#4}}
\end{figure}
}

%Makes a new command for tikz figures with input values: tikz commands, 
% caption and label.
\newcommand{\onetikz}[3]{
\settowidth{\captionwidth}{#1}
\ifthenelse{\lengthtest{\captionwidth<0.7\linewidth}}{\setlength{\captionwidth}{0.7\linewidth}}{}

\begin{figure}
\centering
#1
\centering
\parbox{\linewidth}{\caption{#2}
\label{#3}}
\end{figure}
}

%Makes a new command for two figures next to each other with input values: filename1, caption1, label1,filename2, caption2 and label2. Figure width is set to 0.47\linewidth and the space between the figures is filled with \hfill so the sides of the figures align with to edge of the line.
\newcommand{\twofigure}[6]{
\setlength{\captionwidth}{\linewidth}
\begin{figure*}[ht!]
\begin{minipage}[t]{0.47\linewidth}
\includegraphics[width=\linewidth]{#1}
\centering
\caption{#2}
\label{#3}
\end{minipage}
\hfill
\begin{minipage}[t]{0.47\linewidth}
\centering
\includegraphics[width= \linewidth]{#4}
\centering
\caption{#5}
\label{#6}
\end{minipage}
\end{figure*}
}


%Makes a new command for a table with caption witdh equal to the total table width. Input: tabular, caption and label. Example:
%\onetable{
%\begin{tabular}{ccc}
%a&b&c\\
%\hline
%1&1&1\\
%1&1&1\\
%1&1&1\\
%\end{tabular}
%{The caption.}
%{tab:table1}
%}
\newcommand{\onetable}[3]{
\settowidth{\captionwidth}{#1}
\ifthenelse{\lengthtest{\captionwidth<0.7\linewidth}}{\setlength{\captionwidth}{0.7\linewidth}}{}
\begin{table}
\caption{#2}
\vspace{-0.24cm} %Puts caption close to toprule
\label{#3}
\centering
#1
\end{table}
}

%Makes a long table with captionwidth equal to tablewidth. It takes the following arguments:
%1: Column specifier (e.g. cccc)
%2: Caption
%3: Label
%4: First head (i.e. first row of regular table)
%5: Head of consecutive pages
%6: Foot of pagebreak
%7: Lastfoot (e.g. \midrule)
%8: Body of table
\newcommand{\onelongtable}[8]{
\begin{center}
\settowidth{\captionwidth}{
\begin{tabular}{#1}
#4
#8
\end{tabular}} % This ends the captionwidth part. Next comes the real table.

\begin{longtable}{#1}
\caption{#2}\\
\vspace{-0.74cm} %Puts caption close to toprule
\label{#3}\\

#4
\endfirsthead

#5
\endhead

#6
\endfoot

#7
\endlastfoot

#8
\end{longtable}
\end{center}}




%1:pgfplots code
%2:width
%3:caption
%4:label
\newcommand{\pgfplotsfigure}[4]{
\pgfplotsset{width=#2\linewidth}
\setlength{\captionwidth}{#2\linewidth}
\begin{figure}[t]
\centering
#1
\centering
\parbox{\linewidth}{\caption{#3}
\label{#4}}
\end{figure}
}
